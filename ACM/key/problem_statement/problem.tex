\problemname{The Key to Cryptography}

Suppose you need to encrypt a top secret message like ``SEND MORE MONKEYS".
You could use a simple substitution cipher, where each letter in the alphabet is
replaced with a different letter.  However, these ciphers are easily broken
by using the
fact that certain letters of the alphabet (like `E', `S', and `A') appear more
frequently than others (like `Q', `Z', and `X').  A better encryption scheme would
vary the substitutions used for each letter.  One such system is the
{\it autokey} cipher.

To encrypt a message, you first select a secret word -- say ``ACM" -- and prepend
it to the front of the message.  This longer string is truncated to the length
of the message and called the
{\it key}, and the $n^{th}$ letter of the key is used to encrypt the
$n^{th}$ letter of
the original message.  This encryption is done by treating each letter in the
key as a cyclic shift value for the corresponding letter in the message, where
`A' indicates a shift of 0, `B' a shift of 1, and so on.  Using ``ACM" as the
secret word, we would encrypt our message as follows:
\begin{center}
\begin{tabular}{ll}
\tt SENDMOREMONKEYS & (message) \\
\tt ACMSENDMOREMONK & (key) \\ \hline
\tt SGZVQBUQAFRWSLC & (ciphertext) \\
\end{tabular}
\end{center}
Note that the letter `E' in the message was encrypted as `G' the first time it
was encountered (since the corresponding letter in the key was `C' indicating
a shift of 2), but then as `Q' and `S' the next two times.

Your task is simple: given a ciphertext and the secret word, you must determine
the original message.

\section*{Input}
Input consists of two lines.  The first contains the ciphertext and the
second contains the secret word.  Both lines contain only uppercase
alphabetic letters.

\section*{Output}
Display the original message that generated the given ciphertext using the
given secret word.

